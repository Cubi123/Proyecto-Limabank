% Options for packages loaded elsewhere
\PassOptionsToPackage{unicode}{hyperref}
\PassOptionsToPackage{hyphens}{url}
%
\documentclass[
  ignorenonframetext,
]{beamer}
\usepackage{pgfpages}
\setbeamertemplate{caption}[numbered]
\setbeamertemplate{caption label separator}{: }
\setbeamercolor{caption name}{fg=normal text.fg}
\beamertemplatenavigationsymbolsempty
% Prevent slide breaks in the middle of a paragraph
\widowpenalties 1 10000
\raggedbottom
\setbeamertemplate{part page}{
  \centering
  \begin{beamercolorbox}[sep=16pt,center]{part title}
    \usebeamerfont{part title}\insertpart\par
  \end{beamercolorbox}
}
\setbeamertemplate{section page}{
  \centering
  \begin{beamercolorbox}[sep=12pt,center]{part title}
    \usebeamerfont{section title}\insertsection\par
  \end{beamercolorbox}
}
\setbeamertemplate{subsection page}{
  \centering
  \begin{beamercolorbox}[sep=8pt,center]{part title}
    \usebeamerfont{subsection title}\insertsubsection\par
  \end{beamercolorbox}
}
\AtBeginPart{
  \frame{\partpage}
}
\AtBeginSection{
  \ifbibliography
  \else
    \frame{\sectionpage}
  \fi
}
\AtBeginSubsection{
  \frame{\subsectionpage}
}
\usepackage{lmodern}
\usepackage{amssymb,amsmath}
\usepackage{ifxetex,ifluatex}
\ifnum 0\ifxetex 1\fi\ifluatex 1\fi=0 % if pdftex
  \usepackage[T1]{fontenc}
  \usepackage[utf8]{inputenc}
  \usepackage{textcomp} % provide euro and other symbols
\else % if luatex or xetex
  \usepackage{unicode-math}
  \defaultfontfeatures{Scale=MatchLowercase}
  \defaultfontfeatures[\rmfamily]{Ligatures=TeX,Scale=1}
\fi
\usetheme[]{AnnArbor}
\usecolortheme{dolphin}
\usefonttheme{structurebold}
% Use upquote if available, for straight quotes in verbatim environments
\IfFileExists{upquote.sty}{\usepackage{upquote}}{}
\IfFileExists{microtype.sty}{% use microtype if available
  \usepackage[]{microtype}
  \UseMicrotypeSet[protrusion]{basicmath} % disable protrusion for tt fonts
}{}
\makeatletter
\@ifundefined{KOMAClassName}{% if non-KOMA class
  \IfFileExists{parskip.sty}{%
    \usepackage{parskip}
  }{% else
    \setlength{\parindent}{0pt}
    \setlength{\parskip}{6pt plus 2pt minus 1pt}}
}{% if KOMA class
  \KOMAoptions{parskip=half}}
\makeatother
\usepackage{xcolor}
\IfFileExists{xurl.sty}{\usepackage{xurl}}{} % add URL line breaks if available
\IfFileExists{bookmark.sty}{\usepackage{bookmark}}{\usepackage{hyperref}}
\hypersetup{
  hidelinks,
  pdfcreator={LaTeX via pandoc}}
\urlstyle{same} % disable monospaced font for URLs
\newif\ifbibliography
\usepackage{longtable,booktabs}
\usepackage{caption}
% Make caption package work with longtable
\makeatletter
\def\fnum@table{\tablename~\thetable}
\makeatother
\usepackage{graphicx,grffile}
\makeatletter
\def\maxwidth{\ifdim\Gin@nat@width>\linewidth\linewidth\else\Gin@nat@width\fi}
\def\maxheight{\ifdim\Gin@nat@height>\textheight\textheight\else\Gin@nat@height\fi}
\makeatother
% Scale images if necessary, so that they will not overflow the page
% margins by default, and it is still possible to overwrite the defaults
% using explicit options in \includegraphics[width, height, ...]{}
\setkeys{Gin}{width=\maxwidth,height=\maxheight,keepaspectratio}
% Set default figure placement to htbp
\makeatletter
\def\fps@figure{htbp}
\makeatother
\setlength{\emergencystretch}{3em} % prevent overfull lines
\providecommand{\tightlist}{%
  \setlength{\itemsep}{0pt}\setlength{\parskip}{0pt}}
\setcounter{secnumdepth}{-\maxdimen} % remove section numbering

\author{}
\date{\vspace{-2.5em}}

\begin{document}

\begin{frame}[fragile]{Proyecto-Limabank}
\protect\hypertarget{proyecto-limabank}{}

Proyecto del curso Temas de Investigación de Operaciones 2020-1

\begin{block}{Caso: Agencias}

\begin{block}{Problematica}

Usted ha sido elegido como asesor de un Gerente de agencia del Banco
Limabank. Se requiere la formulacion de una estrategia de planeamiento
para el horizonte de tiempo comprendido entre los meses de abril a junio
de 2019. Para cumplir con esta tarea se ha recopilado informacion de la
agencia desde Enero 2014 hasta Marzo 2019.

\end{block}

\begin{block}{Dataset}

\begin{itemize}
\tightlist
\item
  \textbf{Arribos Totales} = Clientes que llegan a la agencia.\\
\item
  \textbf{T. Espera Total} = Tiempo promedio que espera el cliente en
  cola (minutos).\\
\item
  \textbf{T. Atención Total} = Tiempo promedio que demora un cliente con
  un asesor (minutos).\\
\item
  \textbf{Staff Teórico TM} = Cantidad de asesores en el turno mañana.\\
\item
  \textbf{Staff Teórico TT} = Cantidad de asesores en el turno tarde.\\
\item
  \textbf{Ventas tipo 1} = Cantidad de ventas del producto 1.\\
\item
  \textbf{Ventas tipo 2} = Cantidad de ventas del producto 2.\\
\item
  \textbf{Ventas tipo 3} = Cantidad de ventas del producto 3.\\
\item
  \textbf{Ventas tipo 4} = Cantidad de ventas del producto 4.
\end{itemize}

\end{block}

\begin{block}{Analisis exploratorio de los datos (EDA)}

\begin{longtable}[]{@{}ll@{}}
\caption{Data summary}\tabularnewline
\toprule
\endhead
Name & data\_agencia\tabularnewline
Number of rows & 63\tabularnewline
Number of columns & 9\tabularnewline
\_\_\_\_\_\_\_\_\_\_\_\_\_\_\_\_\_\_\_\_\_\_\_ &\tabularnewline
Column type frequency: &\tabularnewline
numeric & 9\tabularnewline
\_\_\_\_\_\_\_\_\_\_\_\_\_\_\_\_\_\_\_\_\_\_\_\_ &\tabularnewline
Group variables & None\tabularnewline
\bottomrule
\end{longtable}

\textbf{Variable type: numeric}

\begin{longtable}[]{@{}lrrrrrrrrrl@{}}
\toprule
skim\_variable & n\_missing & complete\_rate & mean & sd & p0 & p25 &
p50 & p75 & p100 & hist\tabularnewline
\midrule
\endhead
ArribosTotales & 2 & 0.97 & 5020.33 & 665.09 & 3176.0 & 4623.0 & 5082.0
& 5485.00 & 6217.0 & ▁▃▇▇▅\tabularnewline
StaffTurnoMañana & 0 & 1.00 & 7.51 & 0.67 & 6.0 & 7.0 & 8.0 & 8.00 & 9.0
& ▁▇▁▇▁\tabularnewline
StaffTurnoTarde & 0 & 1.00 & 8.63 & 1.57 & 6.0 & 7.0 & 9.0 & 10.00 &
10.0 & ▂▃▁▂▇\tabularnewline
TiempoAtencion & 0 & 1.00 & 13.73 & 2.40 & 10.8 & 12.1 & 12.8 & 14.45 &
20.0 & ▇▅▁▂▁\tabularnewline
TiempoEspera & 0 & 1.00 & 10.18 & 2.39 & 5.7 & 8.3 & 10.0 & 11.85 & 15.8
& ▃▇▇▃▂\tabularnewline
Ventas\_1 & 0 & 1.00 & 27.97 & 10.14 & 10.0 & 20.0 & 26.0 & 34.50 & 57.0
& ▆▇▆▃▁\tabularnewline
Ventas\_2 & 1 & 0.98 & 82.66 & 49.09 & 27.0 & 47.0 & 63.0 & 132.50 &
186.0 & ▇▃▁▂▂\tabularnewline
Ventas\_3 & 2 & 0.97 & 94.85 & 62.04 & 26.0 & 42.0 & 84.0 & 128.00 &
355.0 & ▇▅▂▁▁\tabularnewline
Ventas\_4 & 0 & 1.00 & 156.79 & 51.68 & 78.0 & 119.0 & 151.0 & 187.00 &
305.0 & ▇▇▇▁▂\tabularnewline
\#\#\# Datos Faltantes & & & & & & & & & &\tabularnewline
\bottomrule
\end{longtable}

Para completar los datos faltantes, se escogio el metodo de
interpolacion. Esto debido a que los datos faltantes son variables
exogenas a la empresa.

A continuacion se pueden ver los datos recopilados por la empresa

\begin{itemize}
\item
  Staff por turno tarde y mañana\\
  \includegraphics{Forecast_files/figure-beamer/unnamed-chunk-3-1.pdf}
\item
  Ventas\\
  \includegraphics{Forecast_files/figure-beamer/unnamed-chunk-4-1.pdf}
\item
  Tiempo espera y atencion
\end{itemize}

\includegraphics{Forecast_files/figure-beamer/unnamed-chunk-5-1.pdf}

\begin{itemize}
\tightlist
\item
  Arribos totales\\
  \includegraphics{Forecast_files/figure-beamer/unnamed-chunk-6-1.pdf}
\end{itemize}

\end{block}

\begin{block}{Pronostico Arribos totales}

Para realizar un planeamiento correcto es necesario conocer cuantos van
a ser los arribos en los siguientes meses, para esto se propone el uso
de dos modelos de series de tiempo (Exponential Triple Smoothing y
Autorregresive Integrated Moving Average) y comparar sus resultados

\textbf{ARIMA}

El ajuste del modelo, nos indica que los datos se ajustan a un modelo
ARIMA(0,1,1)

\begin{verbatim}
## Series: xts_agencia$ArribosTotales 
## ARIMA(0,1,1) 
## 
## Coefficients:
##           ma1
##       -0.3816
## s.e.   0.1210
## 
## sigma^2 estimated as 188860:  log likelihood=-464.16
## AIC=932.32   AICc=932.52   BIC=936.57
## 
## Training set error measures:
##                     ME     RMSE      MAE        MPE     MAPE      MASE
## Training set -16.01653 427.6265 320.1177 -0.9072502 6.681005 0.8655225
##                    ACF1
## Training set 0.02774422
\end{verbatim}

Los errores del modelo cumplen la condicion de normalidad. Tienen un
comportamiento de WN (White Noise). El modelo es valido

\includegraphics{Forecast_files/figure-beamer/unnamed-chunk-8-1.pdf}

\begin{verbatim}
## 
##  Ljung-Box test
## 
## data:  Residuals from ARIMA(0,1,1)
## Q* = 5.8279, df = 9, p-value = 0.757
## 
## Model df: 1.   Total lags used: 10
\end{verbatim}

En el grafico a continuacion se observa el pronostico con intervalos de
confianza de 80\% y 95\%

\includegraphics{Forecast_files/figure-beamer/unnamed-chunk-9-1.pdf}

\textbf{ETS}

El ajuste del modelo, nos indica que los datos se ajustan a un modelo
solo con nivel.

\begin{verbatim}
## ETS(A,N,N) 
## 
## Call:
##  ets(y = xts_agencia$ArribosTotales) 
## 
##   Smoothing parameters:
##     alpha = 0.6116 
## 
##   Initial states:
##     l = 5002.7653 
## 
##   sigma:  434.5695
## 
##      AIC     AICc      BIC 
## 1030.354 1030.761 1036.783 
## 
## Training set error measures:
##                    ME     RMSE      MAE        MPE     MAPE      MASE
## Training set -12.4259 427.6159 321.8933 -0.8399322 6.713345 0.8703233
##                    ACF1
## Training set 0.02593694
\end{verbatim}

Los errores del modelo cumplen la condicion de normalidad. Tienen un
comportamiento de WN (White Noise). El modelo es valido

\includegraphics{Forecast_files/figure-beamer/unnamed-chunk-11-1.pdf}

\begin{verbatim}
## 
##  Ljung-Box test
## 
## data:  Residuals from ETS(A,N,N)
## Q* = 5.3939, df = 8, p-value = 0.7148
## 
## Model df: 2.   Total lags used: 10
\end{verbatim}

En el grafico a continuacion se observa el pronostico con intervalos de
confianza de 80\% y 95\%

\includegraphics{Forecast_files/figure-beamer/unnamed-chunk-12-1.pdf}

Los valores pronosticados por el ajuste de los modelos ARIMA y ETS son
similares, por lo cual es indistinto escoger entre cualquiera de estos
modelos. Para el caso de estudio, elegimos el modelo de ARIMA.

\end{block}

\begin{block}{Modelamiento}

Se tiene como objetivo determinar que variables tienen una mayor
influencia en el tiempo de espera actual. Para conocer la relacion entre
datos se hace uso del grafico de correlaciones (correlograma).

\includegraphics{Forecast_files/figure-beamer/unnamed-chunk-13-1.pdf}

En el correlograma, se puede observar que las variables con mayor
influencia en nuestra variable de impacto, son la cantidad de personas
en el staff del turno tarde y los arribos Totales. Para comprobar esto,
hacemos uso de una anova. Agregamos datos de la serie de tiempo a
nuestro analisis como (mes,año,mitad,trimestre).

\begin{verbatim}
## Analysis of Variance Table
## 
## Response: TiempoEspera
##                  Df  Sum Sq Mean Sq F value    Pr(>F)    
## ArribosTotales    1  51.413  51.413 12.7987 0.0009891 ***
## StaffTurnoMañana  1  13.257  13.257  3.3002 0.0773806 .  
## StaffTurnoTarde   1  31.987  31.987  7.9627 0.0076369 ** 
## TiempoAtencion    1  21.466  21.466  5.3437 0.0264684 *  
## Ventas_1          1   7.683   7.683  1.9127 0.1749585    
## Ventas_2          1   0.109   0.109  0.0272 0.8700158    
## Ventas_3          1   2.056   2.056  0.5119 0.4788045    
## Ventas_4          1   0.324   0.324  0.0806 0.7781299    
## index.num         1   4.433   4.433  1.1036 0.3002893    
## diff              1  13.822  13.822  3.4409 0.0715817 .  
## year              1   7.522   7.522  1.8725 0.1794466    
## year.iso          1   2.203   2.203  0.5483 0.4636713    
## half              1  17.897  17.897  4.4552 0.0416139 *  
## quarter           1   0.210   0.210  0.0522 0.8205186    
## month             1   3.709   3.709  0.9234 0.3428108    
## wday              1   3.614   3.614  0.8997 0.3490082    
## qday              1   0.022   0.022  0.0055 0.9413965    
## yday              1   1.115   1.115  0.2776 0.6013960    
## mweek             1   0.170   0.170  0.0423 0.8381062    
## week              1   9.216   9.216  2.2942 0.1383579    
## week.iso          1   6.513   6.513  1.6214 0.2108454    
## week2             1   1.925   1.925  0.4792 0.4931137    
## week3             1   1.026   1.026  0.2554 0.6162714    
## week4             1   0.015   0.015  0.0038 0.9511583    
## Residuals        37 148.630   4.017                      
## ---
## Signif. codes:  0 '***' 0.001 '**' 0.01 '*' 0.05 '.' 0.1 ' ' 1
\end{verbatim}

De acuerdo al analisis de varianza realizada, las variables con mayor
significacion en el modelo son StaffTurnoTarde, TotalArribos, half
(1:primera mitad del año, 2:segunda mitad del año) y TiempoAtencion.

\includegraphics{Forecast_files/figure-beamer/unnamed-chunk-15-1.pdf}

\textbf{ARIMA}

Ajustamos el modelo de regresion dinamica, modelando los errores con un
modelo de arima.

\begin{verbatim}
## Series: data_arima$TiempoEspera 
## Regression with ARIMA(0,0,0) errors 
## 
## Coefficients:
##       ArribosTotales  StaffTurnoTarde    half
##                9e-04           0.4904  1.0325
## s.e.           3e-04           0.1520  0.4603
## 
## sigma^2 estimated as 4.128:  log likelihood=-132.52
## AIC=273.03   AICc=273.72   BIC=281.61
## 
## Training set error measures:
##                       ME     RMSE      MAE       MPE     MAPE      MASE
## Training set -0.01960623 1.982773 1.633191 -4.157514 16.62299 0.8333976
##                   ACF1
## Training set 0.2276741
\end{verbatim}

\includegraphics{Forecast_files/figure-beamer/unnamed-chunk-17-1.pdf}

\begin{verbatim}
## 
##  Ljung-Box test
## 
## data:  Residuals from Regression with ARIMA(0,0,0) errors
## Q* = 7.8426, df = 7, p-value = 0.3467
## 
## Model df: 3.   Total lags used: 10
\end{verbatim}

Dado que el modelo resultante es un ARIMA(0,0,0), se opta por usar un
modelo lineal.

\textbf{MARS}\\
Debido a que el modelo de ARIMA solo explica un error normal (White
noise) se opta por usar un modelo Multivariado adaptativo de regresion
por splines, con el objetivo de conocer el impacto de las variables en
el modelo.

Se divide el modelo en 80\% y 20\% y se entrena el modelo utilizando
cross validation, 10 folds

\begin{verbatim}
## Call: earth(formula=TiempoEspera~ArribosTotales+StaffTurnoTarde+half,
##             data=data, keepxy=TRUE, nfold=10, ncross=30, varmod.method="lm")
## 
##                        coefficients
## (Intercept)               5.9502256
## half                      1.3863387
## h(ArribosTotales-4837)    0.0200871
## h(ArribosTotales-5077)   -0.0838682
## h(ArribosTotales-5130)    0.0646695
## h(StaffTurnoTarde-7)      0.5607839
## 
## Selected 6 of 13 terms, and 3 of 3 predictors
## Termination condition: Reached nk 21
## Importance: ArribosTotales, StaffTurnoTarde, half
## Number of terms at each degree of interaction: 1 5 (additive model)
## GCV 4.549264  RSS 195.2573  GRSq 0.2184975  RSq 0.4502646  CVRSq -0.480758
## 
## Note: the cross-validation sd's below are standard deviations across folds
## 
## Cross validation:   nterms 3.66 sd 1.42    nvars 2.07 sd 0.89
## 
##      CVRSq   sd     MaxErr sd
##     -0.481 1.37       8.05  4
## 
## varmod: method "lm"    min.sd 0.217    iter.rsq 0.006
## 
## stddev of predictions:
##              coefficients iter.stderr iter.stderr%
## (Intercept)    1.49157887     1.08045           72
## TiempoEspera   0.06681073    0.106399          159
## 
##                               mean   smallest   largest      ratio
## 95% prediction interval   8.514036   7.768278   9.38792   1.208494
## 
##                                          68%    80%    90%    95% 
## response values in prediction interval   76     92     95     98
\end{verbatim}

No se detecta presencia de herestocedasticidad en nuestro modelo.
Algunos outlier en cuanto a los errores residuales. El modelo es valido

\includegraphics{Forecast_files/figure-beamer/unnamed-chunk-19-1.pdf}

Observamos que el numero de bases optimas para la estimacion de nuestro
modelo es 6, en funcion del GRSq

\includegraphics{Forecast_files/figure-beamer/unnamed-chunk-20-1.pdf}

Observamos que las variables mas importantes para la estimacion de
nuestro modelos en orden de mayor a menor importante son los arribos
totales, la cantidad de personas en el staff turno tarde y si nos
encontramos en la primera o segunda mitad del año.

\begin{verbatim}
##                 nsubsets   gcv    rss
## ArribosTotales         5 100.0  100.0
## StaffTurnoTarde        3  45.5   62.1
## half                   2  38.3   50.3
\end{verbatim}

En el siguiente grafico, observamos el impacto que tiene la variacion de
las variables en nuestra variable respuesta Tiempo Espera. En la primera
mitad del año, el tiempo de espera es menor que en la segunda mitad del
año. Asimismo, un aumento en la cantidad de personas en el staff del
turno tarde, se ve reflejado en un aumento en el tiempo espera. Por
ultimo, se observa un impacto mas pronunciado en la variable respuesta,
en el rango de 4837 y 5077. Apartir de 5130 se equilibra el impacto de
esta variable.

\begin{verbatim}
##  plotmo grid:    ArribosTotales StaffTurnoTarde half
##                        5103.066               9    1
\end{verbatim}

\includegraphics{Forecast_files/figure-beamer/unnamed-chunk-22-1.pdf}

Por ultimo, para pronosticar los tiempos de espera para Abril, Mayo,
Junio , se corrieron 1000 simulaciones, utilizando los valores generados
por nuestro ajuste por ARIMA, numero de personas staff tarde igual a 6
(mismo que los meses anteriores) y half igual a 1, ya que nos
encontramos en la primera mitad de año

\includegraphics{Forecast_files/figure-beamer/unnamed-chunk-23-1.pdf}

A traves de los histogramas generados para cada mes, observamos que se
cumple con el objetivo de la gerencia de mantener en 10min el tiempo
limite de espera. La probabilidad de superar ese tiempo de espera para
Abril, Mayo y Junio es de 0.049,0.071,0.063

\end{block}

\end{block}

\end{frame}

\end{document}
